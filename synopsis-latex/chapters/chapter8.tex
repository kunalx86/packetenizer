\chapter{Conclusion}
{
	Active network monitoring is amongst the most efficient network monitoring techniques. Most corporate organizations use Active Network Monitoring to ensure the security and integrity of their networks. This is usually backed by an employee whose job is to examine the output of the Active Monitoring Tool and identify any malicious activity or error. It also requires the employee to have a brief technical knowledge. Due to human intervention there are slight chances of human errors. Packet analysers can be used as an aid in these circumstances. \\ 
	This project will analyse the network traffic and detect possible attacks like DOS/DDOS attempts, unintended connections to SSH, FTP and other services. This will help the user to identify and solve their network problems easily without the need of any technical knowledge. Also, the tedious and time-consuming job of manually examining the monitoring tool is eliminated. Thus, the resulting human resource can be utilized for other important tasks.
}